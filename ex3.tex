% NUMBER 3


The likelihood function summarizes what we know about the parameters of the function. If we write a likelihood function for the number density profile $n(x)$ (Equation (2)), as a product of observations $i$ of parameters we want to explore, and retain terms dependent only on a, b, and c:

\begin{equation}
  L(A,a,b,c \vert x) = \prod_{i} A(a,b,c)\left(\frac{x_i}{b}\right)^{a-3}\exp\left[-\left(\frac{x_i}{b}\right)^c\right]
\end{equation}

The loglikelihood is then $l(A,a,b,c \vert x) = \ln{L(A,a,b,c \vert x)}$

\begin{equation}
  l(A,a,b,c \vert x) = \sum_{i} \ln{A(a,b,c)} + (a-3)\ln{\left(\dfrac{x_i}{b}\right)}-\left(\frac{x_i}{b}\right)^c
\end{equation}



In terms of part (a), the observations we could use to maximize the log-likelihood would be random realizations of points $x_i$ and $n(x_i)$ from the likelihood function. To do this I would use randomly drawn values of a, b, and c, and calculate the A(a,b,c) normalization constant. Then obtaining the x-values from the downloaded data sets, I would have sets of random realizations from which I could maximize the log-likelihood in each data set.


The script used to generate the results is given by:
\lstinputlisting{ex3.py}

It outputs the following results:
\lstinputlisting{ex3output.txt}


An attempt was made at doing part (a) by calculating the loglikehood at many points and finding a maximum.
