% NUMBER 2

The script used to generate the results is given by:

\lstinputlisting{ex2.py}

The result of the script is given by:

It outputs the following results:
\lstinputlisting{ex2output.txt}


\begin{figure}[!htb]
  \centering
  \includegraphics[width=0.8\linewidth]{./plots/interpolationcomparison.png}
  \caption{2(b) Comparing the Lagrange polynomials (via Neville's Algorithm) with Linear Interpolation in the specified range of $x_{max} = 10^{-4}$ to $x_{max} = 5.0$. The top panel is polynomial interpolation of degree M-1 = 4. This introduces extra freedom between the data points which moves away from the log profile. Therefore linear interpolation in the bottom panel captures most of the behavior expect at around the virial radius near $x_{max} = 10^{0}$. }
  \label{fig:interpcompare}
\end{figure}

\begin{figure}[!htb]
  \centering
  \includegraphics[width=0.8\linewidth]{./plots/satellitesof1000haloes.png}
  \caption{2(e) Here 1000 randomly generator haloes have been generated and binned logarithmically to show compare to the expected behavior of the profile dropping off outside the virial radius. We can see the generated data match very well.}
  \label{fig:1000haloes}
\end{figure}


\begin{figure}[!htb]
  \centering
  \includegraphics[width=0.8\linewidth]{./plots/bi-intersection.png}
  \caption{2(f) Half the maximum, y, of the profile is shown as the horizontal dashed line. To find intersection points between these two lines, the profile is shifted down by y/2 so that a bisection algorithm can be called on either side of the maximum, represented by the red and blue triangles. The resulting roots are the yellow stars.}
  \label{fig:intersection}
\end{figure}


\begin{figure}[!htb]
  \centering
  \includegraphics[width=0.8\linewidth]{./plots/poissoncomparison.png}
  \caption{2(g) Taking the largest radial bin from Fig. \ref{fig:1000haloes}, we sort the galaxies falling in that bin by radial distances x. But by taking a histogram of the (area = 1 normalized) data and comparing it to the Poisson pdf generated with 1(a). We can see the result of the natural process of counting galaxies which fall in this radial bin. The mean of the poisson is taken from the mean of the radial bin and generated for events in each bin.}
  \label{fig:comppoisson}
\end{figure}



\begin{figure}[!htb]
  \centering
  \includegraphics[width=0.8\linewidth]{./plots/interpolationfirstdim.png}
  \caption{2(h,\textit{i}) A cube of normalization constant is first generated by evaluating A at 0.1 steps of (a,b,c) in the specified range. THIS cube is what is interpolated over; it essentially will become stretched in each of its 3 dimensions. This first figure shows the result of interpolating in just the first of its three axes. Slices along each axis of the original cube are shown in the left column of images. The right column of images shows slices in the interpolated cube (in just one dimension). Thus, one of its axes has changed its length (doubled, due to setting of my density parameter in code.)}
  \label{fig:onedim}
\end{figure}

\begin{figure}[!htb]
  \centering
  \includegraphics[width=0.8\linewidth]{./plots/interpolationseconddim.png}
  \caption{2(h,\textit{ii}) This figure shows the result of interpolating across the second of its dimensions. Slices along each axis of the original cube are still shown in the left column of images. However, the right column of images now shows slices where two axes have doubled in length. (THIRD axis interpolation not shown due to unsolved bug in code.)}
  \label{fig:twodim}
\end{figure}
